\documentclass{TISE}

% title info
\title{Not just normal: exploring power with Shiny apps}
\author{Christian Stratton and Dr. Jennifer L. Green \\ Montana State University}

\begin{document}
	
\section{INTRODUCTION}

\begin{itemize}
	\item motivate need for app
	\begin{itemize}
		\item learning objectives
		\item highlight how visuals lead to deeper understanding of power
		\item model after \cite{mcdaniel2012}
	\end{itemize}
	\item lit review
	\begin{itemize}
		\item cite ASA guidelines
		\item GAISE guidelines for intro stat
	\end{itemize}
	\item CONNECTIONS
\end{itemize}

\section{MOTIVATION}

\begin{itemize}
	\item describe power teaching experience with Christian's cohort
	\item discuss features of the app that address difficulties introduced by example
	\begin{itemize}
		\item highlight versatility
	\end{itemize}
\end{itemize}

\textbf{Features:}
\begin{itemize}
	\item Showcase power functions and sampling distributions of test statistics under a wide range of conditions without working through time consuming derivations.
	\item Have derivations available to work through.
	\item Requires no coding experience
\end{itemize}

\section{APPLET IMPLEMENTATION}

\textbf{HOW IT COULD BE USED}

VERSATILITY -> HERE ARE THREE POSSIBLE WAYS:

\begin{itemize}
	\item describe how to use it, structured around learning objectives
	\item highlight versatility for different levels of instruction
\end{itemize}


\textbf{Possible topics for exploration:}
\begin{itemize}
	\item Explore power curves and understand how these curves change for different tests statistics, null values, alternative hypotheses, alpha levels, sample sizes, and population distributions. \textbf{START WITH COMPARING POP DISTRIBUTIONS THEN RESTRICT TO ONE POP DIST AND DISCUSS OTHER STUFF}
	\begin{itemize}
		\item sufficiency
		\item Provides a high level of versatility in what students can explore with the app (power curves, sampling distributions, sufficient statistics, derivations, simulation, etc.) and \textbf{connections}
		\item ex 1 - sum of exponential, ex 2 - order stats normal and ex 3 - uniform ????
	\end{itemize}

	
	\item Explore how power curves arise by relating them to the sampling distribution of the test statistic. 
	\item Understand difficulties of determining power in real life situations
	\begin{itemize}
		\item Investigate alternatives to determining power such as normal approximations and simulation
	\end{itemize}
\end{itemize}

\section{IMPLEMENTATION}

\textbf{DISCUSS HOW IT WAS USED WITH REAL LIFE HUMANS AND WHAT THEY GOT FROM IT}

MENTION 422, HOW IT WAS USED (JUST EXP) -> FOCUS ON 502 HERE -> "MORE SCAFFOLDING" FOR 422

LOGISTICS -> DAY 1, 2, 3 ETC -> FOR GRAD STUDENTS



\section{FUTURE WORK AND CONCLUSIONS}



\newpage
\bibliography{references}

\end{document}