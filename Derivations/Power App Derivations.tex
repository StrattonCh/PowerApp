\documentclass[]{article}
\RequirePackage{amssymb, amsfonts, amsmath, latexsym, verbatim, xspace, setspace}
\RequirePackage{tikz, pgflibraryplotmarks}

\usepackage[margin=1in]{geometry}
\usepackage{amsmath}
\usepackage{amssymb}
\usepackage{subcaption}
\usepackage{float}
\usepackage{listings}
\usepackage{kbordermatrix}
\lstset{
	basicstyle=\ttfamily,
	columns=fullflexible,
	frame=single,
	breaklines=true,
	postbreak=\mbox{\textcolor{red}{$\hookrightarrow$}\space},
	keepspaces = TRUE,
}

\newcommand{\bfX}{\mbox{\boldmath $X$}}
\newcommand{\bfy}{\mbox{\boldmath $y$}}
\newcommand{\bfB}{\mbox{\boldmath $\beta$}}
\newcommand{\bfe}{\mbox{\boldmath $\epsilon$}}
\newcommand{\bfZ}{\mbox{\boldmath $Z$}}
\newcommand{\bfBt}{\mbox{$\tilde{\boldsymbol{\beta}}$}}
\providecommand{\floor}[1]{\left \lfloor #1 \right \rfloor }

% Here's where you edit the Class, Exam, Date, etc.
\newcommand{\class}{STAT 578}
\newcommand{\examnum}{Homework 6}
\newcommand{\examdate}{due 03/01/18}

\singlespacing
% \onehalfspacing
% \doublespacing

\parindent 0ex
\title{Power App Derivations}
\date{}

\begin{document}
	
\maketitle

\section{Exponential Distribution}

\subsection{Statistic: $\sum_{i=1}^n X_i$}

\subsubsection{Alternative: Greater than}

Suppose that $X_i \stackrel{iid}{\sim} \text{Exp}(\theta)$, and we take a random sample of size $n$ from this population. Further suppose that we wish to test the following hypotheses:
\[
\begin{split}
\text{$H_0$: }& \theta \leq \theta_0 \\
\text{$H_a$: }& \theta > \theta_0
\end{split}
\]

Consider the following test function:
\[
\phi(\bfX) = \begin{cases}
1 & \sum_{i=1}^n X_i > k \\
0 & \text{else}
\end{cases},
\]

where $k$ is chosen such that $Pr(\phi(\bfX) = 1|\theta_0) = \alpha$. It can be shown that if $X_i \stackrel{iid}{\sim} \text{Exp}(\theta)$, then $\sum_{i=1}^{n} X_i \sim \text{Gamma}(n, \theta)$. Using this relationship, we derive the test. 
\[
\begin{split}
Pr(\phi(\bfX) = 1|\theta_0) = Pr\left(\sum_{i=1}^n X_i > k\Big|\theta_0\right) = \alpha 
\end{split}
\]

Let $T = \sum_{i=1}^n X_i \sim \text{Gamma}(n, \theta)$. To define our test, we seek the value $k$ such that 
\[
Pr(T > k | \theta_0) = 1 - Pr(T \leq k | \theta_0) = \alpha
\]

The value of $k$ that satisfies this equation is the $(1 - \alpha)^{th}$ quantile of the $\text{Gamma}(n, \theta_0)$ distribution. Letting $\Gamma_{n, \theta_0, 1-\alpha}$ denote this value, our test functions becomes
\[
\phi(\bfX) = \begin{cases}
1 & \sum_{i=1}^n X_i > \Gamma_{n, \theta_0, 1-\alpha} \\
0 & \text{else}
\end{cases}
\]

Therefore, our power function is
\[
\beta(\theta) = Pr(\bfX \in RR) = Pr\left(\sum_{i=1}^n X_i > \Gamma_{n, \theta_0, 1-\alpha} \Big| \theta\right) = 1 - Pr\left(\sum_{i=1}^n X_i \leq \Gamma_{n, \theta_0, 1-\alpha} \Big| \theta\right)
\]

\subsubsection{Alternative: Less than}

Suppose that $X_i \stackrel{iid}{\sim} \text{Exp}(\theta)$, and we take a random sample of size $n$ from this population. Further suppose that we wish to test the following hypotheses:
\[
\begin{split}
\text{$H_0$: }& \theta \geq \theta_0 \\
\text{$H_a$: }& \theta < \theta_0
\end{split}
\]

Consider the following test function:
\[
\phi(\bfX) = \begin{cases}
1 & \sum_{i=1}^n X_i < k \\
0 & \text{else}
\end{cases},
\]

where $k$ is chosen such that $Pr(\phi(\bfX) = 1|\theta_0) = \alpha$. Let $T = \sum_{i=1}^n X_i \sim \text{Gamma}(n, \theta)$. To define our test, we seek the value $k$ such that 
\[
Pr(T < k | \theta_0) = \alpha
\]

The value of $k$ that satisfies this equation is the $\alpha^{th}$ quantile of the $\text{Gamma}{n, \theta_0}$ distribution. Letting $\Gamma_{n, \theta_0, \alpha}$ denote this value, our test functions becomes
\[
\phi(\bfX) = \begin{cases}
1 & \sum_{i=1}^n X_i < \Gamma_{n, \theta_0, \alpha} \\
0 & \text{else}
\end{cases}
\]

Therefore, our power function is
\[
\beta(\theta) = Pr(\bfX \in RR) = Pr\left(\sum_{i=1}^n X_i < \Gamma_{n, \theta_0, \alpha} \Big| \theta\right) 
\]

\subsubsection{Alternative: Not equal to}

Suppose that $X_i \stackrel{iid}{\sim} \text{Exp}(\theta)$, and we take a random sample of size $n$ from this population. Further suppose that we wish to test the following hypotheses:
\[
\begin{split}
\text{$H_0$: }& \theta = \theta_0 \\
\text{$H_a$: }& \theta \neq \theta_0
\end{split}
\]

Consider the following test function:
\[
\phi(\bfX) = \begin{cases}
1 & \sum_{i=1}^n X_i < k_1 \text{ or } \sum_{i=1}^n X_i > k_2 \\
0 & \text{else}
\end{cases},
\]

where $k_1$ and $k_2$ are chosen such that $Pr(\phi(\bfX) = 1|\theta_0) = \alpha$. In this case, we assume a symmetric test function in that $k_1$ and $k_2$ are chosen such that $Pr(\sum_{i=1}^n X_i < k_1) = \alpha/2$ and $Pr(\sum_{i=1}^n X_i > k_2) = \alpha/2$. Let $T = \sum_{i=1}^n X_i \sim \text{Gamma}(n, \theta)$. 

The values of $k_1$ and $k_2$ that satisfy these equations are the $(\alpha/2)^{th}$ and $(1 - \alpha/2)^{th}$ quantiles of the $\text{Gamma}{n, \theta_0}$ distribution. Letting $\Gamma_{n, \theta_0, \alpha/2}$ and $\Gamma_{n, \theta_0, 1- \alpha/2}$ denote these values, our test functions becomes
\[
\phi(\bfX) = \begin{cases}
1 & \sum_{i=1}^n X_i < \Gamma_{n, \theta_0, \alpha/2} \text{ or } \sum_{i=1}^n X_i > \Gamma_{n, \theta_0, 1 - \alpha/2}\\
0 & \text{else}
\end{cases}
\]

Therefore, our power function is
\[
\beta(\theta) = Pr(\bfX \in RR) = 1 - Pr\left(\sum_{i=1}^n X_i < \Gamma_{n, \theta_0, 1 - \alpha/2}\right) +  Pr\left(\sum_{i=1}^n X_i > \Gamma_{n, \theta_0, \alpha/2} \Big| \theta\right) 
\]

\subsection{Statistic: $X_{(1)}$}

\subsubsection{Alternative: Greater than}

Suppose that $X_i \stackrel{iid}{\sim} \text{Exp}(\theta)$, and we take a random sample of size $n$ from this population. Further suppose that we wish to test the following hypotheses:
\[
\begin{split}
\text{$H_0$: }& \theta \leq \theta_0 \\
\text{$H_a$: }& \theta > \theta_0
\end{split}
\]

Consider the following test function:
\[
\phi(\bfX) = \begin{cases}
1 & X_{(1)} > k \\
0 & \text{else}
\end{cases},
\]

where $k$ is chosen such that $Pr(\phi(\bfX) = 1|\theta_0) = \alpha$. It can be shown that if $X_i \stackrel{iid}{\sim} \text{Exp}(\theta)$, then $X_{(1)} \sim \text{Exp}(\frac{\theta}{n})$. Using this relationship, we derive the test. 
\[
\begin{split}
Pr(\phi(\bfX) = 1|\theta_0) = Pr\left(X_{(1)} > k\Big|\theta_0\right) = \alpha 
\end{split}
\]

Let $T = X_{(1)} \sim \text{Exp}(\frac{\theta}{n})$. To define our test, we seek the value $k$ such that 
\[
Pr(T > k | \theta_0) = 1 - Pr(T \leq k | \theta_0) = \alpha
\]

The value of $k$ that satisfies this equation is the $(1 - \alpha)^{th}$ quantile of the $\text{Exp}(\frac{\theta_0}{n})$ distribution. Letting $\eta_{\theta_0/n, 1-\alpha}$ denote this value, our test functions becomes
\[
\phi(\bfX) = \begin{cases}
1 & X_{(1)} > \eta_{\theta_0/n, 1-\alpha} \\
0 & \text{else}
\end{cases}
\]

Therefore, our power function is
\[
\beta(\theta) = Pr(\bfX \in RR) = Pr\left(X_{(1)} > \eta_{\theta_0/n, 1-\alpha} \Big| \theta\right) = 1 - Pr\left(X_{(1)}\leq \eta_{\theta_0/n, 1-\alpha} \Big| \theta\right)
\]

\subsubsection{Alternative: Less than}

Suppose that $X_i \stackrel{iid}{\sim} \text{Exp}(\theta)$, and we take a random sample of size $n$ from this population. Further suppose that we wish to test the following hypotheses:
\[
\begin{split}
\text{$H_0$: }& \theta \geq \theta_0 \\
\text{$H_a$: }& \theta < \theta_0
\end{split}
\]

Consider the following test function:
\[
\phi(\bfX) = \begin{cases}
1 & X_{(1)} < k \\
0 & \text{else}
\end{cases},
\]

where $k$ is chosen such that $Pr(\phi(\bfX) = 1|\theta_0) = \alpha$. Let $T = X_{(1)} \sim \text{Exp}(\frac{\theta}{n})$. To define our test, we seek the value $k$ such that 
\[
Pr(T < k | \theta_0) = \alpha
\]

The value of $k$ that satisfies this equation is the $\alpha^{th}$ quantile of the $\text{Exp}(\frac{\theta_0}{n})$ distribution. Letting $\eta_{\theta_0/n, \alpha}$ denote this value, our test functions becomes
\[
\phi(\bfX) = \begin{cases}
1 & X_{(1)} < \eta_{\theta_0/n, \alpha} \\
0 & \text{else}
\end{cases}
\]

Therefore, our power function is
\[
\beta(\theta) = Pr(\bfX \in RR) = Pr\left(X_{(1)} < \eta_{\theta_0/n, \alpha} \Big| \theta\right)
\]

\subsubsection{Alternative: Not equal to}

Suppose that $X_i \stackrel{iid}{\sim} \text{Exp}(\theta)$, and we take a random sample of size $n$ from this population. Further suppose that we wish to test the following hypotheses:
\[
\begin{split}
\text{$H_0$: }& \theta = \theta_0 \\
\text{$H_a$: }& \theta \neq \theta_0
\end{split}
\]

Consider the following test function:
\[
\phi(\bfX) = \begin{cases}
1 & X_{(1)} < k_1 \text{ or } X_{(1)} > k_2 \\
0 & \text{else}
\end{cases},
\]

where $k_1$ and $k_2$ are chosen such that $Pr(\phi(\bfX) = 1|\theta_0) = \alpha$. In this case, we choose $k_1$ and $k_2$ such that our test is symmetric. Let $T = X_{(1)} \sim \text{Exp}(\frac{\theta}{n})$. To define our test, we seek the values $k_1$ and $k_2$ such that $Pr(T < k_1 | \theta_0) = \alpha/2$ and $Pr(T > k_2 | \theta_0) = \alpha/2$. 

The values of $k_1$ and $k_2$ that satisfy these equations are the $(\alpha/2)^{th}$ and $(1 - \alpha/2)^{th}$ quantiles of the $\text{Exp}(\frac{\theta_0}{n})$ distribution. Letting $\eta_{\theta_0/n, \alpha/2}$ and $\eta_{\theta_0/n, 1 - \alpha/2}$ denote this value, our test functions becomes
\[
\phi(\bfX) = \begin{cases}
1 & X_{(1)} < \eta_{\theta_0/n, \alpha/2} \text{ or } X_{(1)} > \eta_{\theta_0/n, 1 - \alpha/2}\\
0 & \text{else}
\end{cases}
\]

Therefore, our power function is
\[
\beta(\theta) = Pr(\bfX \in RR) = 1 - Pr\left(X_{(1)} < \eta_{\theta_0/n, 1 - \alpha/2}\right) + Pr\left(X_{(1)} < \eta_{\theta_0/n, \alpha/2} \Big| \theta\right)
\]

\subsection{Statistic: $X_{(n)}$}

\subsubsection{Alternative: Greater than}

Suppose that $X_i \stackrel{iid}{\sim} \text{Exp}(\theta)$, and we take a random sample of size $n$ from this population. Further suppose that we wish to test the following hypotheses:
\[
\begin{split}
\text{$H_0$: }& \theta \leq \theta_0 \\
\text{$H_a$: }& \theta > \theta_0
\end{split}
\]

Consider the following test function:
\[
\phi(\bfX) = \begin{cases}
1 & X_{(n)} > k \\
0 & \text{else}
\end{cases},
\]

where $k$ is chosen such that $Pr(\phi(\bfX) = 1|\theta_0) = \alpha$. To find this value of $k$, we need to know the distribution of the sample max. By definition, 
\[
F_{X_{(n)}}(x) = \left[F_X(x)\right]^n = \left[1 - \text{exp}(-\frac{x}{\theta})\right]^n
\]


Let $T = X_{(n)}$. To define our test, we seek the value $k$ such that 
\[
Pr(T > k | \theta_0) = 1 - Pr(T \leq k | \theta_0) = \alpha
\]

Using the distribution function derived above, we have
\[
\begin{split}
1 - \left[1 - \text{exp}(-\frac{k}{\theta_0})\right]^n &= \alpha \\
\left[1 - \text{exp}(-\frac{k}{\theta_0})\right]^n &= 1 - \alpha \\
1 - \text{exp}(-\frac{k}{\theta_0}) &= (1-\alpha)^{1/n} \\
\text{exp}(-\frac{k}{\theta_0}) &= 1 - (1-\alpha)^{1/n} \\
-\frac{k}{\theta_0} &= \log(1 - (1-\alpha)^{1/n}) \\
k &= -\theta_0\log(1 - (1-\alpha)^{1/n})
\end{split}
\]

Therefore, our test function is
\[
\phi(\bfX) = \begin{cases}
1 & X_{(n)} > -\theta_0\log(1 - (1-\alpha)^{1/n}) \\
0 & \text{else}
\end{cases}
\]

Therefore, our power function is
\[
\begin{split}
\beta(\theta) &= Pr(\bfX \in RR) = Pr\left(X_{(n)} > -\theta_0\log(1 - (1-\alpha)^{1/n}) \Big| \theta\right) \\
&= 1 - Pr\left(X_{(n)}\leq -\theta_0\log(1 - (1-\alpha)^{1/n}) \Big| \theta\right) \\
&= 1 - \left[1 - \text{exp}\left(\frac{-\theta_0\log(1 - (1-\alpha)^{1/n})}{\theta}\right)\right]^n
\end{split}
\]


\subsubsection{Alternative: Less than}

Suppose that $X_i \stackrel{iid}{\sim} \text{Exp}(\theta)$, and we take a random sample of size $n$ from this population. Further suppose that we wish to test the following hypotheses:
\[
\begin{split}
\text{$H_0$: }& \theta \geq \theta_0 \\
\text{$H_a$: }& \theta < \theta_0
\end{split}
\]

Consider the following test function:
\[
\phi(\bfX) = \begin{cases}
1 & X_{(n)} < k \\
0 & \text{else}
\end{cases},
\]

where $k$ is chosen such that $Pr(\phi(\bfX) = 1|\theta_0) = \alpha$. Let $T = X_{(n)}$. To define our test, we seek the value $k$ such that 
\[
Pr(T < k | \theta_0) = \alpha
\]

Using the distribution function derived above, we have
\[
\begin{split}
\left[1 - \text{exp}(-\frac{k}{\theta_0})\right]^n &= \alpha \\
1 - \text{exp}(-\frac{k}{\theta_0}) &= \alpha^{1/n} \\
\text{exp}(-\frac{k}{\theta_0}) &= 1 - \alpha^{1/n} \\
-\frac{k}{\theta_0} &= \log(1 - \alpha^{1/n}) \\
k &= -\theta_0\log(1 - \alpha^{1/n})
\end{split}
\]

Therefore, our test function is
\[
\phi(\bfX) = \begin{cases}
1 & X_{(n)} < -\theta_0\log(1 - \alpha^{1/n}) \\
0 & \text{else}
\end{cases}
\]

Therefore, our power function is
\[
\begin{split}
\beta(\theta) &= Pr(\bfX \in RR) = Pr\left(X_{(n)} < -\theta_0\log(1 - \alpha^{1/n}) \Big| \theta\right) \\
&= Pr\left(X_{(n)} < -\theta_0\log(1 - \alpha^{1/n}) \Big| \theta\right) \\
&= \left[1 - \text{exp}\left(\frac{-\theta_0\log(1 - \alpha^{1/n})}{\theta}\right)\right]^n
\end{split}
\]

\subsubsection{Alternative: Not equal to}

Suppose that $X_i \stackrel{iid}{\sim} \text{Exp}(\theta)$, and we take a random sample of size $n$ from this population. Further suppose that we wish to test the following hypotheses:
\[
\begin{split}
\text{$H_0$: }& \theta = \theta_0 \\
\text{$H_a$: }& \theta \neq \theta_0
\end{split}
\]

Consider the following test function:
\[
\phi(\bfX) = \begin{cases}
1 & X_{(n)} < k_1 \text{ or } X_{(n)} > k_2 \\
0 & \text{else}
\end{cases},
\]

where $k_1$ and $k_2$ are chosen such that $Pr(\phi(\bfX) = 1|\theta_0) = \alpha$. In this case, we choose a symmetric test function. Let $T = X_{(n)}$. To define our test, we seek the values $k_1$ and $k_2$ such that $Pr(T < k_1 | \theta_0) = \alpha/2$ and $Pr(T > k_2 | \theta_0) = \alpha/2$. Using the distribution function derived above, we have
\[
\begin{split}
\left[1 - \text{exp}(-\frac{k_1}{\theta_0})\right]^n &= \alpha/2 \\
1 - \text{exp}(-\frac{k_1}{\theta_0}) &= (\alpha/2)^{1/n} \\
\text{exp}(-\frac{k_1}{\theta_0}) &= 1 - (\alpha/2)^{1/n} \\
-\frac{k_1}{\theta_0} &= \log(1 - (\alpha/2)^{1/n}) \\
k_1 &= -\theta_0\log(1 - (\alpha/2)^{1/n})
\end{split}
\]

Similarly, 
\[
\begin{split}
1 - \left[1 - \text{exp}(-\frac{k_2}{\theta_0})\right]^n &= \alpha/2 \\
\left[1 - \text{exp}(-\frac{k_2}{\theta_0})\right]^n &= 1 - \alpha/2 \\
1 - \text{exp}(-\frac{k_2}{\theta_0}) &= (1-\alpha/2)^{1/n} \\
\text{exp}(-\frac{k_2}{\theta_0}) &= 1 - (1-\alpha/2)^{1/n} \\
-\frac{k_2}{\theta_0} &= \log(1 - (1-\alpha/2)^{1/n}) \\
k_2 &= -\theta_0\log(1 - (1-\alpha/2)^{1/n})
\end{split}
\]


Therefore, our test function is
\[
\phi(\bfX) = \begin{cases}
1 & X_{(n)} < -\theta_0\log(1 - (\alpha/2)^{1/n}) \text{ or } X_{(n)} > -\theta_0\log(1 - (1-\alpha/2)^{1/n}) \\
0 & \text{else}
\end{cases}
\]

Therefore, our power function is
\[
\begin{split}
\beta(\theta) &= Pr(\bfX \in RR)  \\
&= 1 - Pr\left(X_{(n)} < -\theta_0\log(1 - (1-\alpha/2)^{1/n} \Big| \theta\right) + Pr\left(X_{(n)} < -\theta_0\log(1 - (\alpha/2)^{1/n}) \Big| \theta\right)\\
&= 1 - \left[1 - \text{exp}\left(\frac{-\theta_0\log(1 - (1-\alpha/2)^{1/n})}{\theta}\right)\right]^n + \left[1 - \text{exp}\left(\frac{-\theta_0\log(1 - (\alpha/2)^{1/n})}{\theta}\right)\right]^n
\end{split}
\]

\pagebreak

\section{Normal Distribution}

\subsection{Statistic: $\sum_{i=1}^n X_i$}

\subsubsection{Alternative: Greater than}

Suppose that $X_i \stackrel{iid}{\sim} \text{N}(\theta, \sigma^2)$, where $\sigma^2$ is known, and we take a random sample of size $n$ from this population. Further suppose that we wish to test the following hypotheses:
\[
\begin{split}
\text{$H_0$: }& \theta \leq \theta_0 \\
\text{$H_a$: }& \theta > \theta_0
\end{split}
\]

Consider the following test function:
\[
\phi(\bfX) = \begin{cases}
1 & \sum_{i=1}^n X_i > k \\
0 & \text{else}
\end{cases},
\]

where $k$ is chosen such that $Pr(\phi(\bfX) = 1|\theta_0) = \alpha$. It can be shown that if $X_i \stackrel{iid}{\sim} \text{N}(\theta, \sigma^2)$, then $\sum_{i=1}^{n} X_i \sim \text{N}(n\theta, n\sigma^2)$. Using this relationship, we derive the test. 
\[
\begin{split}
Pr(\phi(\bfX) = 1|\theta_0) = Pr\left(\sum_{i=1}^n X_i > k\Big|\theta_0\right) = \alpha 
\end{split}
\]

Let $T = \sum_{i=1}^{n} X_i \sim \text{N}(n\theta, n\sigma^2)$. To define our test, we seek the value $k$ such that 
\[
Pr(T > k | \theta_0) = 1 - Pr(T \leq k | \theta_0) = \alpha
\]

The value of $k$ that satisfies this equation is the $(1 - \alpha)^{th}$ quantile of the $\text{N}(n\theta_0, n\sigma^2)$ distribution. Letting $z^*_{n\theta_0, n\sigma^2, 1-\alpha}$ denote this value, our test functions becomes
\[
\phi(\bfX) = \begin{cases}
1 & \sum_{i=1}^n X_i > z^*_{n\theta_0, n\sigma^2, 1-\alpha} \\
0 & \text{else}
\end{cases}
\]

Therefore, our power function is
\[
\beta(\theta) = Pr(\bfX \in RR) = Pr\left(\sum_{i=1}^n X_i > z^*_{n\theta_0, n\sigma^2, 1-\alpha} \Big| \theta\right) = 1 - Pr\left(\sum_{i=1}^n X_i \leq z^*_{n\theta_0, n\sigma^2, 1-\alpha}  \Big| \theta\right)
\]

\textbf{\textcolor{blue}{Derive in terms of typical Z test?}}

\subsubsection{Alternative: Less than}

Suppose that $X_i \stackrel{iid}{\sim} \text{N}(\theta, \sigma^2)$, where $\sigma^2$ is known, and we take a random sample of size $n$ from this population. Further suppose that we wish to test the following hypotheses:
\[
\begin{split}
\text{$H_0$: }& \theta \geq \theta_0 \\
\text{$H_a$: }& \theta < \theta_0
\end{split}
\]

Consider the following test function:
\[
\phi(\bfX) = \begin{cases}
1 & \sum_{i=1}^n X_i < k \\
0 & \text{else}
\end{cases},
\]

where $k$ is chosen such that $Pr(\phi(\bfX) = 1|\theta_0) = \alpha$. Let $T = \sum_{i=1}^{n} X_i \sim \text{N}(n\theta, n\sigma^2)$. To define our test, we seek the value $k$ such that 
\[
Pr(T < k | \theta_0) = \alpha
\]

The value of $k$ that satisfies this equation is the $\alpha^{th}$ quantile of the $\text{N}(n\theta_0, n\sigma^2)$ distribution. Letting $z^*_{n\theta_0, n\sigma^2, \alpha}$ denote this value, our test function becomes
\[
\phi(\bfX) = \begin{cases}
1 & \sum_{i=1}^n X_i < z^*_{n\theta_0, n\sigma^2, \alpha} \\
0 & \text{else}
\end{cases}
\]

Therefore, our power function is
\[
\beta(\theta) = Pr(\bfX \in RR) = Pr\left(\sum_{i=1}^n X_i < z^*_{n\theta_0, n\sigma^2, \alpha} \Big| \theta\right) 
\]

\textbf{\textcolor{blue}{Derive in terms of typical Z test?}}

\subsubsection{Alternative: Not equal to}

Suppose that $X_i \stackrel{iid}{\sim} \text{N}(\theta, \sigma^2)$, where $\sigma^2$ is known, and we take a random sample of size $n$ from this population. Further suppose that we wish to test the following hypotheses:
\[
\begin{split}
\text{$H_0$: }& \theta = \theta_0 \\
\text{$H_a$: }& \theta \neq \theta_0
\end{split}
\]

Consider the following test function:
\[
\phi(\bfX) = \begin{cases}
1 & \sum_{i=1}^n X_i > |k| \\
0 & \text{else}
\end{cases},
\]

where $k$ are is chosen such that $Pr(\phi(\bfX) = 1|\theta_0) = \alpha$. Note that we assume a symmetric test function in that $k$ is chosen such that $Pr(\sum_{i=1}^n X_i < -k) = \alpha/2$ and $Pr(\sum_{i=1}^n X_i > k) = \alpha/2$. Let $T = \sum_{i=1}^n X_i \sim \text{N}(n\theta, n\sigma^2)$. The value of $k$ that satisfies this equation is the $(1-\alpha/2)^{th}$ quantile of the $\text{N}{n\theta_0, n\sigma^2}$ distribution. Letting $z^*_{n\theta_0, n\sigma^2, 1-\alpha/2}$ denote this value, our test function becomes
\[
\phi(\bfX) = \begin{cases}
1 & \sum_{i=1}^n X_i < z^*_{n\theta_0, n\sigma^2, \alpha/2} \text{ or } \sum_{i=1}^n X_i > z^*_{n\theta_0, n\sigma^2, 1-\alpha/2} \\
0 & \text{else}
\end{cases}
\]

Therefore, our power function is
\[
\beta(\theta) = Pr(\bfX \in RR) = 1 - Pr\left(\sum_{i=1}^n X_i < z^*_{n\theta_0, n\sigma^2, 1-\alpha/2}\right) +  Pr\left(\sum_{i=1}^n X_i > -z^*_{n\theta_0, n\sigma^2, 1-\alpha/2} \Big| \theta\right) 
\]


\subsection{Statistic: $X_{(1)}$}

\subsubsection{Alternative: Greater than}

Suppose that $X_i \stackrel{iid}{\sim} \text{N}(\theta, \sigma^2)$, where $\sigma^2$ is known, and we take a random sample of size $n$ from this population. Further suppose that we wish to test the following hypotheses:
\[
\begin{split}
\text{$H_0$: }& \theta \leq \theta_0 \\
\text{$H_a$: }& \theta > \theta_0
\end{split}
\]

Consider the following test function:
\[
\phi(\bfX) = \begin{cases}
1 & X_{(1)} > k \\
0 & \text{else}
\end{cases},
\]

where $k$ is chosen such that $Pr(\phi(\bfX) = 1|\theta_0) = \alpha$. To find this value, we must first find the distribution of the sample minimum. By definition,
\[
F_{X_{(1)}}(x) = 1 - [1 - F_X(x)]^n = 1 - [1 - \psi(x, \theta_0, \sigma^2)]^n,
\]
where $\psi(x, \theta_0, \sigma^2)$ denotes the distribution function of the $N(\theta_0, \sigma^2)$ distribution. Let $T = X_{(1)}$. We seek the value $k$ such that 
\[
Pr(T > k | \theta_0) = 1 - Pr(T \leq k | \theta_0) = \alpha
\]

Using the distribution function derived above, we have:
\[
\begin{split}
1 - Pr(T \leq k | \theta_0) = 1 - \left(1 - [1 - \psi(k, \theta_0, \sigma^2)]^n\right) &= \alpha \\
[1 - \psi(k, \theta_0, \sigma^2)]^n &= \alpha \\
\psi(k, \theta_0, \sigma^2) &= 1 - \alpha^{1/n}
\end{split}
\]

The value of $k$ that satisfies this equation is the $(1 - \alpha^{1/n})^{th}$ quantile of the $\text{N}(\theta_0, \sigma^2)$ distribution. Letting $z^*_{\theta_0, \sigma^2, 1-\alpha^{1/n}}$ denote this value, our test functions becomes
\[
\phi(\bfX) = \begin{cases}
1 & X_{(1)} > z^*_{\theta_0, \sigma^2, 1-\alpha^{1/n}} \\
0 & \text{else}
\end{cases}
\]

Therefore, our power function is
\[
\beta(\theta) = Pr(\bfX \in RR) = Pr\left(X_{(1)} > z^*_{\theta_0, \sigma^2, 1-\alpha^{1/n}}\Big| \theta\right) = 1 - Pr\left(X_{(1)}\leq z^*_{\theta_0, \sigma^2, 1-\alpha^{1/n}} \Big| \theta\right)
\]

\subsubsection{Alternative: Less than}

Suppose that $X_i \stackrel{iid}{\sim} \text{N}(\theta, \sigma^2)$, where $\sigma^2$ is known, and we take a random sample of size $n$ from this population. Further suppose that we wish to test the following hypotheses:
\[
\begin{split}
\text{$H_0$: }& \theta \geq \theta_0 \\
\text{$H_a$: }& \theta < \theta_0
\end{split}
\]

Consider the following test function:
\[
\phi(\bfX) = \begin{cases}
1 & X_{(1)} < k \\
0 & \text{else}
\end{cases},
\]

where $k$ is chosen such that $Pr(\phi(\bfX) = 1|\theta_0) = \alpha$. Let $T = X_{(1)}$. We seek the value $k$ such that 
\[
Pr(T < k | \theta_0) = \alpha
\]

Using the distribution function derived above, we have:
\[
\begin{split}
Pr(T < k | \theta_0) = 1 - [1 - \psi(k, \theta_0, \sigma^2)]^n &= \alpha \\
[1 - \psi(k, \theta_0, \sigma^2)]^n &= 1- \alpha \\
\psi(k, \theta_0, \sigma^2) &= 1 - (1 - \alpha)^{1/n}
\end{split}
\]

The value of $k$ that satisfies this equation is the $1 - (1 - \alpha)^{1/n}$ quantile of the $\text{N}(\theta_0, \sigma^2)$ distribution. Letting $z^*_{\theta_0, \sigma^2, 1 - (1-\alpha)^{1/n}}$ denote this value, our test functions becomes
\[
\phi(\bfX) = \begin{cases}
1 & X_{(1)} < z^*_{\theta_0, \sigma^2, 1 - (1-\alpha)^{1/n}} \\
0 & \text{else}
\end{cases}
\]

Therefore, our power function is
\[
\beta(\theta) = Pr(\bfX \in RR) = Pr\left(X_{(1)} < z^*_{\theta_0, \sigma^2, 1 - (1-\alpha)^{1/n}}\Big| \theta\right)
\]

\subsubsection{Alternative: Not equal to}

Suppose that $X_i \stackrel{iid}{\sim} \text{N}(\theta, \sigma^2)$, where $\sigma^2$ is known, and we take a random sample of size $n$ from this population. Further suppose that we wish to test the following hypotheses:
\[
\begin{split}
\text{$H_0$: }& \theta = \theta_0 \\
\text{$H_a$: }& \theta \neq \theta_0
\end{split}
\]

Consider the following test function:
\[
\phi(\bfX) = \begin{cases}
1 & X_{(1)} < k_1 \text{ or } X_{(1)} > k_2 \\
0 & \text{else}
\end{cases},
\]

where $k_1$ and $k_2$ are chosen such that $Pr(\phi(\bfX) = 1|\theta_0) = \alpha$. We choose a symmetric test. Let $T = X_{(1)}$. We seek the values $k_1$ and $k_2$ such that $Pr(T < k_1 | \theta_0) = \alpha/2$ and $Pr(T > k_2 | \theta_0) = \alpha/2$. Using the distribution function derived above, we have:
\[
\begin{split}
Pr(T < k_1 | \theta_0) = 1 - [1 - \psi(k_1, \theta_0, \sigma^2)]^n &= \alpha/2 \\
[1 - \psi(k_1, \theta_0, \sigma^2)]^n &= 1- \alpha/2 \\
\psi(k_1, \theta_0, \sigma^2) &= 1 - (1 - \alpha/2)^{1/n}
\end{split}
\]

The value of $k_1$ that satisfies this equation is the $1 - (1 - \alpha/2)^{1/n}$ quantile of the $\text{N}(\theta_0, \sigma^2)$ distribution. We let $z^*_{\theta_0, \sigma^2, 1 - (1-\alpha/2)^{1/n}}$ denote this value. Similarly,
\[
\begin{split}
1 - Pr(T \leq k_2 | \theta_0) = 1 - \left(1 - [1 - \psi(k_2, \theta_0, \sigma^2)]^n\right) &= \alpha/2 \\
[1 - \psi(k_2, \theta_0, \sigma^2)]^n &= \alpha/2 \\
\psi(k_2, \theta_0, \sigma^2) &= 1 - (\alpha/2)^{1/n}
\end{split}
\]
The value of $k_2$ that satisfies this equation is the $1 - (\alpha/2)^{1/n}$ quantile of the $\text{N}(\theta_0, \sigma^2)$ distribution. We let $z^*_{\theta_0, \sigma^2, 1 - (\alpha/2)^{1/n}}$ denote this value. Therefore, our test function becomes
\[
\phi(\bfX) = \begin{cases}
1 & X_{(1)} < z^*_{\theta_0, \sigma^2, 1 - (1-\alpha/2)^{1/n}} \text{ or } X_{(1)} > z^*_{\theta_0, \sigma^2, 1 - (\alpha/2)^{1/n}}\\
0 & \text{else}
\end{cases}
\]

Therefore, our power function is
\[
\beta(\theta) = Pr(\bfX \in RR) = 1 - Pr\left(X_{(1)} < z^*_{\theta_0, \sigma^2, 1 - (\alpha/2)^{1/n}} \Big| \theta\right) + Pr\left(X_{(1)} < z^*_{\theta_0, \sigma^2, 1 - (1-\alpha/2)^{1/n}} \Big| \theta\right)
\]

\subsection{Statistic: $X_{(n)}$}

\subsubsection{Alternative: Greater than}

Suppose that $X_i \stackrel{iid}{\sim} \text{N}(\theta, \sigma^2)$, where $\sigma^2$ is known, and we take a random sample of size $n$ from this population. Further suppose that we wish to test the following hypotheses:
\[
\begin{split}
\text{$H_0$: }& \theta \leq \theta_0 \\
\text{$H_a$: }& \theta > \theta_0
\end{split}
\]

Consider the following test function:
\[
\phi(\bfX) = \begin{cases}
1 & X_{(n)} > k \\
0 & \text{else}
\end{cases},
\]

where $k$ is chosen such that $Pr(\phi(\bfX) = 1|\theta_0) = \alpha$. To find this value, we must first find the distribution of the sample maximum. By definition,
\[
F_{X_{(n)}}(x) = [F_X(x)]^n = [\psi(x, \theta_0, \sigma^2)]^n,
\]
where $\psi(x, \theta_0, \sigma^2)$ denotes the distribution function of the $N(\theta_0, \sigma^2)$ distribution. Let $T = X_{(n)}$. We seek the value $k$ such that 
\[
Pr(T > k | \theta_0) = 1 - Pr(T \leq k | \theta_0) = \alpha
\]

Using the distribution function derived above, we have:
\[
\begin{split}
1 - Pr(T \leq k | \theta_0) = 1 - [\psi(k, \theta_0, \sigma^2)]^n &= \alpha \\
\psi(k, \theta_0, \sigma^2) &= (1 - \alpha)^{1/n}
\end{split}
\]

The value of $k$ that satisfies this equation is the $(1 - \alpha)^{1/n}$ quantile of the $\text{N}(\theta_0, \sigma^2)$ distribution. Letting $z^*_{\theta_0, \sigma^2, (1 - \alpha)^{1/n}}$ denote this value, our test functions becomes
\[
\phi(\bfX) = \begin{cases}
1 & X_{(n)} > z^*_{\theta_0, \sigma^2, (1 - \alpha)^{1/n}} \\
0 & \text{else}
\end{cases}
\]

Therefore, our power function is
\[
\beta(\theta) = Pr(\bfX \in RR) = Pr\left(X_{(n)} > z^*_{\theta_0, \sigma^2, (1 - \alpha)^{1/n}}\Big| \theta\right) = 1 - Pr\left(X_{(n)}\leq z^*_{\theta_0, \sigma^2, (1 - \alpha)^{1/n}} \Big| \theta\right)
\]

\subsubsection{Alternative: Less than}

Suppose that $X_i \stackrel{iid}{\sim} \text{N}(\theta, \sigma^2)$, where $\sigma^2$ is known, and we take a random sample of size $n$ from this population. Further suppose that we wish to test the following hypotheses:
\[
\begin{split}
\text{$H_0$: }& \theta \geq \theta_0 \\
\text{$H_a$: }& \theta < \theta_0
\end{split}
\]

Consider the following test function:
\[
\phi(\bfX) = \begin{cases}
1 & X_{(n)} < k \\
0 & \text{else}
\end{cases},
\]

where $k$ is chosen such that $Pr(\phi(\bfX) = 1|\theta_0) = \alpha$. Let $T = X_{(n)}$. We seek the value $k$ such that 
\[
Pr(T < k | \theta_0) = \alpha
\]

Using the distribution function derived above, we have:
\[
\begin{split}
Pr(T < k | \theta_0) = [\psi(k, \theta_0, \sigma^2)]^n &= \alpha \\
\psi(k, \theta_0, \sigma^2) &= \alpha^{1/n}
\end{split}
\]

The value of $k$ that satisfies this equation is the $\alpha^{1/n}$ quantile of the $\text{N}(\theta_0, \sigma^2)$ distribution. Letting $z^*_{\theta_0, \sigma^2, \alpha^{1/n}}$ denote this value, our test functions becomes
\[
\phi(\bfX) = \begin{cases}
1 & X_{(n)} > z^*_{\theta_0, \sigma^2, \alpha^{1/n}} \\
0 & \text{else}
\end{cases}
\]

Therefore, our power function is
\[
\beta(\theta) = Pr(\bfX \in RR) = Pr\left(X_{(n)} < z^*_{\theta_0, \sigma^2, \alpha^{1/n}}\Big| \theta\right) 
\]

\subsubsection{Alternative: Not equal to}

Suppose that $X_i \stackrel{iid}{\sim} \text{N}(\theta, \sigma^2)$, where $\sigma^2$ is known, and we take a random sample of size $n$ from this population. Further suppose that we wish to test the following hypotheses:
\[
\begin{split}
\text{$H_0$: }& \theta = \theta_0 \\
\text{$H_a$: }& \theta \neq \theta_0
\end{split}
\]

Consider the following test function:
\[
\phi(\bfX) = \begin{cases}
1 & X_{(n)} < k_1 \text{ or } X_{(n)} > k_2 \\
0 & \text{else}
\end{cases},
\]

where $k_1$ and $k_2$ are chosen such that $Pr(\phi(\bfX) = 1|\theta_0) = \alpha$. We choose a symmetric test. Let $T = X_{(n)}$. We seek the values $k_1$ and $k_2$ such that $Pr(T < k_1 | \theta_0) = \alpha/2$ and $Pr(T > k_2 | \theta_0) = \alpha/2$. Using the distribution function derived above, we have:
\[
\begin{split}
Pr(T < k_1 | \theta_0) = [\psi(k_1, \theta_0, \sigma^2)]^n &= \alpha/2 \\
\psi(k_1, \theta_0, \sigma^2) &= (\alpha/2)^{1/n}
\end{split}
\]

The value of $k_1$ that satisfies this equation is the $(\alpha/2)^{1/n}$ quantile of the $\text{N}(\theta_0, \sigma^2)$ distribution. We let $z^*_{\theta_0, \sigma^2, (\alpha/2)^{1/n}}$ denote this value. Similarly,
\[
\begin{split}
1 - Pr(T \leq k_2 | \theta_0) = 1 - [\psi(k_2, \theta_0, \sigma^2)]^n&= \alpha/2 \\
[\psi(k_2, \theta_0, \sigma^2)]^n &= 1 - \alpha/2 \\
\psi(k_2, \theta_0, \sigma^2) &= (1 - \alpha/2)^{1/n}
\end{split}
\]
The value of $k_2$ that satisfies this equation is the $(1 - \alpha/2)^{1/n}$ quantile of the $\text{N}(\theta_0, \sigma^2)$ distribution. We let $z^*_{\theta_0, \sigma^2, (1 - \alpha/2)^{1/n}}$ denote this value. Therefore, our test function becomes
\[
\phi(\bfX) = \begin{cases}
1 & X_{(n)} < z^*_{\theta_0, \sigma^2, (\alpha/2)^{1/n}} \text{ or } X_{(n)} > z^*_{\theta_0, \sigma^2, (1 - \alpha/2)^{1/n}}\\
0 & \text{else}
\end{cases}
\]

Therefore, our power function is
\[
\beta(\theta) = Pr(\bfX \in RR) = 1 - Pr\left(X_{(n)} < z^*_{\theta_0, \sigma^2, (1 - \alpha/2)^{1/n}} \Big| \theta\right) + Pr\left(X_{(n)} < z^*_{\theta_0, \sigma^2, (\alpha/2)^{1/n}} \Big| \theta\right)
\]

\section{Uniform Distribution}

\subsection{Statistic: $\sum_{i=1}^n X_i$}

\subsubsection{Extending the Irwin-Hall Distribution}

It can be shown that if $X_i \stackrel{iid}{\sim} \text{Unif}(0, 1)$ and we take a random sample of size $n$ from this population, then $\sum_{i=1}^n X_i \sim \text{Irwin-Hall}(n)$. Suppose that $Y \sim \sim \text{Irwin-Hall}(n)$, then $Y$ has the following density and distribution functions:
\[
f_Y(y) =\begin{cases}
\frac{1}{(n-1)!} \sum_{k = 0}^{\floor{x}} (-1)^k \binom{n}{k}(y-k)^{n-1}, \ y \in \mathbb{R} & 0 < y < n\theta \\
0 & \text{else}
\end{cases} 
\]

\[
F_Y(y) = \begin{cases}
0 & y < 0 \\
\frac{1}{n!} \sum_{k = 0}^{\floor{x}} (-1)^k \binom{n}{k}(y-k)^{n} & 0 \leq y \leq n\theta \\
1 & y > n\theta 
\end{cases}
\]

We desire the distribution of the sum of $n$ uniform(0, $\theta$) random variables. It can be shown that if $W_i \stackrel{iid}{\sim} \text{Unif}(0, \theta)$, then $\sum_{i=1}^n W_i$ has the same distribution as $\theta X$, where $X \sim \text{Irwin-Hall}(n)$. That is, let $T = \sum_{i=1}^n W_i$, then
\[
Pr(T \leq t) = Pr(\theta X \leq t) = Pr\left(X \leq \frac{t}{\theta}\right)
\]

Therefore, 
\[
\begin{split}
F_T(t) &= F_Y\left(\frac{t}{\theta} \right) \\
f_T(t) &= \frac{1}{\theta} f_Y\left(\frac{t}{\theta}\right)
\end{split}
\]

\subsubsection{Alternative: Greater than}

Suppose that $X_i \stackrel{iid}{\sim} \text{Unif}(0, \theta)$, and we take a random sample of size $n$ from this population. Further suppose that we wish to test the following hypotheses:
\[
\begin{split}
\text{$H_0$: }& \theta \leq \theta_0 \\
\text{$H_a$: }& \theta > \theta_0
\end{split}
\]

Consider the following test function:
\[
\phi(\bfX) = \begin{cases}
1 & \sum_{i=1}^n X_i > k \\
0 & \text{else}
\end{cases},
\]

where $k$ is chosen such that $Pr(\phi(\bfX) = 1|\theta_0) = \alpha$. That is, we seek $k$ such that
\[
Pr(\phi(\bfX) = 1|\theta_0) = Pr\left(\sum_{i=1}^n X_i > k |\theta_0\right) = \alpha
\]

The value of $k$ that satisfies this is the $1 - \alpha$ quantile of the general Irwin-Hall($n, \theta_0$) distribution. Letting $\eta_{1-\alpha, n, \theta_0}$ denote this quantity, our test function is

\[
\phi(\bfX) = \begin{cases}
1 & \sum_{i=1}^n X_i > \eta_{1-\alpha, n, \theta_0}\\
0 & \text{else}
\end{cases}
\]

Therefore, our power function looks like
\[
\begin{split}
\beta(\theta) &= Pr(\bfX \in RR) = Pr(\sum_{i=1}^n X_i > \eta_{1-\alpha, n, \theta}) \\
&= 1 - Pr\left(\sum_{i=1}^n X_i \leq \eta_{1-\alpha, n, \theta_0}\right)
\end{split}
\]

\subsubsection{Alternative: Less than}

Suppose that $X_i \stackrel{iid}{\sim} \text{Unif}(0, \theta)$, and we take a random sample of size $n$ from this population. Further suppose that we wish to test the following hypotheses:
\[
\begin{split}
\text{$H_0$: }& \theta \geq \theta_0 \\
\text{$H_a$: }& \theta < \theta_0
\end{split}
\]

Consider the following test function:
\[
\phi(\bfX) = \begin{cases}
1 & \sum_{i=1}^n X_i < k \\
0 & \text{else}
\end{cases},
\]

where $k$ is chosen such that $Pr(\phi(\bfX) = 1|\theta_0) = \alpha$. That is, we seek $k$ such that
\[
Pr(\phi(\bfX) = 1|\theta_0) = Pr\left(\sum_{i=1}^n X_i < k |\theta_0\right) = \alpha
\]

The value of $k$ that satisfies this is the $\alpha$ quantile of the general Irwin-Hall($n, \theta_0$) distribution. Letting $\eta_{\alpha, n, \theta_0}$ denote this quantity, our test function is

\[
\phi(\bfX) = \begin{cases}
1 & \sum_{i=1}^n X_i < \eta_{\alpha, n, \theta_0}\\
0 & \text{else}
\end{cases}
\]

Therefore, our power function looks like
\[
\begin{split}
\beta(\theta) &= Pr(\bfX \in RR) = Pr\left(\sum_{i=1}^n X_i < \eta_{\alpha, n, \theta_0}\right)
\end{split}
\]


\subsubsection{Alternative: Not equal to}

Suppose that $X_i \stackrel{iid}{\sim} \text{Unif}(0, \theta)$, and we take a random sample of size $n$ from this population. Further suppose that we wish to test the following hypotheses:
\[
\begin{split}
\text{$H_0$: }& \theta = \theta_0 \\
\text{$H_a$: }& \theta \neq \theta_0
\end{split}
\]

Consider the following test function:
\[
\phi(\bfX) = \begin{cases}
1 & \sum_{i=1}^n X_i < k_1 \text{ or } \sum_{i=1}^n X_i > k_2 \\
0 & \text{else}
\end{cases},
\]

where $k_1$ and $k_2$ are chosen such that $Pr(\phi(\bfX) = 1|\theta_0) = \alpha$. We choose a symmetric test. Let $T = \sum_{i=1}^n X_i$. We seek $k_1$ and $k_2$ such that $Pr(T < k_1|\theta_0) = \alpha/2$ and $Pr(T < k_2|\theta_0) = \alpha/2$. The values of $k_1$ and $k_2$ that satisfy these equations are the $\alpha/2$ and $1 - \alpha/2$ quantiles of the general Irwin-Hall($n, \theta_0$) distribution. Letting $\eta_{\alpha/2, n, \theta_0}$ and $\eta_{\alpha/2, n, \theta_0}$ denote these quantities, our test function is

\[
\phi(\bfX) = \begin{cases}
1 & \sum_{i=1}^n X_i < \eta_{\alpha, n, \theta}\\
0 & \text{else}
\end{cases}
\]

Therefore, our power function looks like
\[
\begin{split}
\beta(\theta) &= Pr(\bfX \in RR) = 1 - Pr\left(\sum_{i=1}^n X_i < \eta_{1-\alpha/2, n, \theta_0}\right) + Pr\left(\sum_{i=1}^n X_i < \eta_{\alpha/2, n, \theta_0}\right) \\
\end{split}
\]

\subsection{Statistic: $X_{(1)}$}

\subsubsection{Alternative: Greater than}

Suppose that $X_i \stackrel{iid}{\sim} \text{Unif}(0, \theta)$, and we take a random sample of size $n$ from this population. Further suppose that we wish to test the following hypotheses:
\[
\begin{split}
\text{$H_0$: }& \theta \leq \theta_0 \\
\text{$H_a$: }& \theta > \theta_0
\end{split}
\]

Consider the following test function:
\[
\phi(\bfX) = \begin{cases}
1 & X_{(1)} > k \\
0 & \text{else}
\end{cases},
\]

where $k$ is chosen such that $Pr(\phi(\bfX) = 1|\theta_0) = \alpha$. To find this value of $k$, we need to know the distribution of the sample min. By definition, 
\[
F_{X_{(1)}}(x) = 1 - \left[1 - F_X(x)\right]^n = 1 - \left[1 - \frac{x}{\theta}\right]^n
\]

Let $T = X_{(1)}$. To define our test, we seek the value $k$ such that 
\[
Pr(T > k | \theta_0) = 1 - Pr(T \leq k | \theta_0) = \alpha
\]

Using the distribution function derived above, we have
\[
\begin{split}
\left[1 - \frac{k}{\theta_0}\right]^n &= \alpha \\
1 - \frac{k}{\theta_0} &= \alpha^{1/n} \\
k &= \theta_0(1 - \alpha^{1/n})
\end{split}
\]

Therefore, our test function is
\[
\phi(\bfX) = \begin{cases}
1 & X_{(1)} > \theta_0(1-\alpha^{1/n}) \\
0 & \text{else}
\end{cases}
\]

Therefore, our power function is piecewise and looks like the following:\\
\textbf{For $\theta \leq k$:}
\[
\beta(\theta) = 0
\]
\textbf{For $\theta > k$:}
\[
\begin{split}
\beta(\theta) &= Pr(\bfX \in RR) = Pr\left(X_{(1)} > \theta_0(1-\alpha^{1/n}) \Big| \theta\right) \\
&= 1 - Pr\left(X_{(1)}\leq \theta_0(1-\alpha^{1/n}) \Big| \theta\right) \\
&= \left[1 - \frac{\theta_0(1-\alpha^{1/n})}{\theta}\right]^n
\end{split}
\]

\subsubsection{Alternative: Less than}

Suppose that $X_i \stackrel{iid}{\sim} \text{Unif}(0, \theta)$, and we take a random sample of size $n$ from this population. Further suppose that we wish to test the following hypotheses:
\[
\begin{split}
\text{$H_0$: }& \theta \geq \theta_0 \\
\text{$H_a$: }& \theta < \theta_0
\end{split}
\]

Consider the following test function:
\[
\phi(\bfX) = \begin{cases}
1 & X_{(1)} < k \\
0 & \text{else}
\end{cases},
\]

where $k$ is chosen such that $Pr(\phi(\bfX) = 1|\theta_0) = \alpha$. Let $T = X_{(1)}$. To define our test, we seek the value $k$ such that 
\[
Pr(T < k | \theta_0) = \alpha
\]

Using the distribution function derived above, we have
\[
\begin{split}
1 - \left[1 - \frac{k}{\theta_0}\right]^n &= \alpha \\
1 - \frac{k}{\theta_0} &= (1 - \alpha)^{1/n} \\
k &= \theta_0(1 - (1 - \alpha)^{1/n})
\end{split}
\]

Therefore, our test function is
\[
\phi(\bfX) = \begin{cases}
1 & X_{(1)} < \theta_0(1 - (1 - \alpha)^{1/n}) \\
0 & \text{else}
\end{cases}
\]

Therefore, our power function is piecewise and looks like the following:\\
\textbf{For $\theta \leq k$:}
\[
\beta(\theta) = 1
\]
\textbf{For $\theta > k$:}
\[
\begin{split}
\beta(\theta) &= Pr(\bfX \in RR) = Pr\left(X_{(1)} < \theta_0(1 - (1 - \alpha)^{1/n}) \Big| \theta\right) \\
&= 1 - \left[1 - \frac{\theta_0(1 - (1 - \alpha)^{1/n})}{\theta}\right]^n
\end{split}
\]

\subsubsection{Alternative: Not equal to}

Suppose that $X_i \stackrel{iid}{\sim} \text{Unif}(0, \theta)$, and we take a random sample of size $n$ from this population. Further suppose that we wish to test the following hypotheses:
\[
\begin{split}
\text{$H_0$: }& \theta = \theta_0 \\
\text{$H_a$: }& \theta \neq \theta_0
\end{split}
\]

Consider the following test function:
\[
\phi(\bfX) = \begin{cases}
1 & X_{(1)} < k_1 \text{ or } X_{(1)} > k_2 \\
0 & \text{else}
\end{cases},
\]

where $k_1$ and $k_2$ are chosen such that $Pr(\phi(\bfX) = 1|\theta_0) = \alpha$. We choose a symmetric test here. Let $T = X_{(1)}$. To define our test, we seek the values $k_1$ and $k_2$ such that $Pr(T < k_1) = \alpha/2$ and $Pr(T > k_2) = \alpha/2$.

Using the distribution function derived above, we have
\[
\begin{split}
1 - \left[1 - \frac{k_1}{\theta_0}\right]^n &= \alpha/2 \\
\left[1 - \frac{k_1}{\theta_0}\right]^n &= 1 - \alpha/2 \\
1 - \frac{k_1}{\theta_0} &= (1 - \alpha/2)^{1/n} \\
\frac{k_1}{\theta_0} &= 1 - (1 - \alpha/2)^{1/n} \\
k_1 &= \theta_0(1 - (1 - \alpha/2)^{1/n}) \\
\end{split}
\]

Similarly, 
\[
\begin{split}
\left[1 - \frac{k_2}{\theta_0}\right]^n &= \alpha/2 \\
1 - \frac{k_2}{\theta_0} &= (\alpha/2)^{1/n} \\
\frac{k_2}{\theta_0} &= 1 - (\alpha/2)^{1/n} \\
k_2 &= \theta_0(1 - (\alpha/2)^{1/n})
\end{split}
\]

Therefore, our test function is
\[
\phi(\bfX) = \begin{cases}
1 & X_{(1)} < \theta_0(1 - (1 - \alpha/2)^{1/n}) \text{ or } X_{(1)} >  \theta_0(1 - (\alpha/2)^{1/n}) \\
0 & \text{else}
\end{cases}
\]

Therefore, our power function is piecewise and looks like:\\
\textbf{For $0 < \theta \leq k_1$}
\[
\beta(\theta) = 1
\]
\textbf{For $k_1 < \theta \leq k_2$}
\[
\begin{split}
\beta(\theta) &= Pr(\bfX \in RR) = Pr\left(X_{(1)} < \theta_0(1 - (1 - \alpha/2)^{1/n}) \Big| \theta\right) \\
&= 1 - \left[1 - \frac{\theta_0(1 - (1 - \alpha/2)^{1/n}) }{\theta}\right]^n
\end{split}
\]
\textbf{For $\theta > k_2$}
\[
\begin{split}
\beta(\theta) &= Pr(\bfX \in RR) = Pr\left(X_{(1)} > \theta_0(1 - (\alpha/2)^{1/n}) \Big| \theta\right) + Pr\left(X_{(1)} < \theta_0(1 - (1 - \alpha/2)^{1/n}) \Big| \theta\right) \\
&= 1 - Pr\left(X_{(n)}\leq  \theta_0(1 - (\alpha/2)^{1/n}) \Big| \theta\right) + Pr\left(X_{(n)} < \theta_0(1 - (1 - \alpha/2)^{1/n}) \Big| \theta\right) \\
&= 1 + \left[1 - \frac{\theta_0(1 - (\alpha/2)^{1/n})}{\theta}\right]^n - \left[1 - \frac{\theta_0(1 - (1 - \alpha/2)^{1/n})}{\theta}\right]^n
\end{split}
\]

\subsection{Statistic: $X_{(n)}$}

\subsubsection{Alternative: Greater than}

Suppose that $X_i \stackrel{iid}{\sim} \text{Unif}(0, \theta)$, and we take a random sample of size $n$ from this population. Further suppose that we wish to test the following hypotheses:
\[
\begin{split}
\text{$H_0$: }& \theta \leq \theta_0 \\
\text{$H_a$: }& \theta > \theta_0
\end{split}
\]

Consider the following test function:
\[
\phi(\bfX) = \begin{cases}
1 & X_{(n)} > k \\
0 & \text{else}
\end{cases},
\]

where $k$ is chosen such that $Pr(\phi(\bfX) = 1|\theta_0) = \alpha$. To find this value of $k$, we need to know the distribution of the sample max. By definition, 
\[
F_{X_{(n)}}(x) = \left[F_X(x)\right]^n = \left[\frac{x}{\theta}\right]^n = \frac{x^n}{\theta^n}
\]


Let $T = X_{(n)}$. To define our test, we seek the value $k$ such that 
\[
Pr(T > k | \theta_0) = 1 - Pr(T \leq k | \theta_0) = \alpha
\]

Using the distribution function derived above, we have
\[
\begin{split}
1 - \frac{k^n}{\theta_0^n} &= \alpha \\
\frac{k^n}{\theta_0^n} &= 1 - \alpha \\
k^n &= \theta_0^n(1-\alpha) \\
k &= \theta_0(1-\alpha)^{1/n}
\end{split}
\]

Therefore, our test function is
\[
\phi(\bfX) = \begin{cases}
1 & X_{(n)} > \theta_0(1-\alpha)^{1/n} \\
0 & \text{else}
\end{cases}
\]

Therefore, our power function is piecewise and looks like the following:\\
\textbf{For $0 < \theta < k$:}
\[
\beta(\theta) = 0
\]
\textbf{For $\theta > k$:}
\[
\begin{split}
\beta(\theta) &= Pr(\bfX \in RR) = Pr\left(X_{(n)} > \theta_0(1-\alpha)^{1/n} \Big| \theta\right) \\
&= 1 - Pr\left(X_{(n)}\leq \theta_0(1-\alpha)^{1/n} \Big| \theta\right) \\
&= 1 - \frac{\theta_0^n(1-\alpha)}{\theta^n}
\end{split}
\]

\subsubsection{Alternative: Less than}

Suppose that $X_i \stackrel{iid}{\sim} \text{Unif}(0, \theta)$, and we take a random sample of size $n$ from this population. Further suppose that we wish to test the following hypotheses:
\[
\begin{split}
\text{$H_0$: }& \theta \geq \theta_0 \\
\text{$H_a$: }& \theta < \theta_0
\end{split}
\]

Consider the following test function:
\[
\phi(\bfX) = \begin{cases}
1 & X_{(n)} < k \\
0 & \text{else}
\end{cases},
\]

where $k$ is chosen such that $Pr(\phi(\bfX) = 1|\theta_0) = \alpha$. Let $T = X_{(n)}$. To define our test, we seek the value $k$ such that 
\[
Pr(T < k | \theta_0) = \alpha
\]

Using the distribution function derived above, we have
\[
\begin{split}
\frac{k^n}{\theta_0^n} &= \alpha \\
k^n &= \theta_0^n\alpha \\
k &= \theta_0\alpha^{1/n}
\end{split}
\]

Therefore, our test function is
\[
\phi(\bfX) = \begin{cases}
1 & X_{(n)} < \theta_0\alpha^{1/n}  \\
0 & \text{else}
\end{cases}
\]

Therefore, our power function is piecewise and looks like:\\
\textbf{For $0 < \theta < k$}
\[
\beta(\theta) = 1
\]
\textbf{For $\theta \geq k$}
\[
\begin{split}
\beta(\theta) &= Pr(\bfX \in RR) = Pr\left(X_{(n)} < \theta_0\alpha^{1/n} \Big| \theta\right) \\
&= \frac{\theta_0^n\alpha}{\theta^n}
\end{split}
\]

\subsubsection{Alternative: Not equal to}

Suppose that $X_i \stackrel{iid}{\sim} \text{Unif}(0, \theta)$, and we take a random sample of size $n$ from this population. Further suppose that we wish to test the following hypotheses:
\[
\begin{split}
\text{$H_0$: }& \theta = \theta_0 \\
\text{$H_a$: }& \theta \neq \theta_0
\end{split}
\]

Consider the following test function:
\[
\phi(\bfX) = \begin{cases}
1 & X_{(n)} < k_1 \text{ or } X_{(n)} > k_2 \\
0 & \text{else}
\end{cases},
\]

where $k_1$ and $k_2$ are chosen such that $Pr(\phi(\bfX) = 1|\theta_0) = \alpha$. We choose a symmetric test here. Let $T = X_{(n)}$. To define our test, we seek the values $k_1$ and $k_2$ such that $Pr(T < k_1) = \alpha/2$ and $Pr(T > k_2) = \alpha/2$.

Using the distribution function derived above, we have
\[
\begin{split}
\frac{k_1^n}{\theta_0^n} &= \alpha/2 \\
k_1^n &= \theta_0^n\alpha/2 \\
k_1 &= \theta_0(\alpha/2)^{1/n}
\end{split}
\]

Similarly, 
\[
\begin{split}
1 - \frac{k_2^n}{\theta_0^n} &= \alpha/2 \\
\frac{k^n}{\theta_0^n} &= 1 - \alpha/2 \\
k_2^n &= \theta_0^n(1-\alpha/2) \\
k_2 &= \theta_0(1-\alpha/2)^{1/n}
\end{split}
\]

Therefore, our test function is
\[
\phi(\bfX) = \begin{cases}
1 & X_{(n)} < \theta_0(\alpha/2)^{1/n} \text{ or } X_{(n)} >  \theta_0(1-\alpha/2)^{1/n} \\
0 & \text{else}
\end{cases}
\]

Therefore, our power function is piecewise and looks like:\\
\textbf{For $0 < \theta \leq k_1$}
\[
\beta(\theta) = 1
\]
\textbf{For $k_1 < \theta \leq k_2$}
\[
\begin{split}
\beta(\theta) &= Pr(\bfX \in RR) = Pr\left(X_{(n)} < \theta_0(\alpha/2)^{1/n} \Big| \theta\right) \\
&= \frac{\theta_0^n\alpha}{2\theta^n}
\end{split}
\]
\textbf{For $\theta > k_2$}
\[
\begin{split}
\beta(\theta) &= Pr(\bfX \in RR) = Pr\left(X_{(n)} > \theta_0(1-\alpha/2)^{1/n} \Big| \theta\right) + Pr\left(X_{(n)} < \theta_0(\alpha/2)^{1/n} \Big| \theta\right) \\
&= 1 - Pr\left(X_{(n)}\leq \theta_0(1-\alpha/2)^{1/n} \Big| \theta\right) + Pr\left(X_{(n)} < \theta_0(\alpha/2)^{1/n} \Big| \theta\right) \\
&= 1 - \frac{\theta_0^n(1-\alpha/2)}{\theta^n} + \frac{\theta_0^n\alpha}{2\theta^n}
\end{split}
\]

\textcolor{blue}{\textbf{SAME AS SAMPLE MIN ?!?!?! ?!?!?!}}

\end{document}
